\documentclass{article}

% PACKAGES
\usepackage[english]{babel}
\usepackage[letterpaper,top=2cm,bottom=2cm,left=3cm,right=3cm,marginparwidth=1.75cm]{geometry}
\usepackage{amsmath}
\usepackage{graphicx}
\usepackage[colorlinks=true, allcolors=blue]{hyperref}

% Title
\title{CHAPTER 9: Model for puffs and slugs in pulsatile pipe flow}
\date{}
%%%%%%%%%%%%%%%%%%%%%%%%%%%%%%%%%%%%%%%%%%%%%%%%%%%%%%%%%%%%%%%%%%%%%%%%%%%%%%%%%%%%%%%%%%%%%%%%
% BEGIN DOCUMENT
%%%%%%%%%%%%%%%%%%%%%%%%%%%%%%%%%%%%%%%%%%%%%%%%%%%%%%%%%%%%%%%%%%%%%%%%%%%%%%%%%%%%%%%%%%%%%%%%
\begin{document}
\maketitle
%%%%%%%%%%%%%%%%%%%%%%%%%%%%%%%%%%%%%%%%%%%%%%%%%%%%%%%%%%%%%%%%%%
% Introduction: 
% - Motivation to develop an extension of the model: 
%   · We observe puffs and slugs in pulsatile pipe flow...
%   · Pulsatile pipe flow influenced by many features...
% - We first explain the extension of the model 
% - We show some results of the model, compared with DNS, justify model decisions
% - We talk about limitations and future prospects


%%%%%%%%%%%%%%%%%%%%%%%%%%%%%%%%%%%%%%%%%%%%%%%%%%%%%%%%%
% Section: the model
% - We justify the changes... (include maybe results without those changes?)
% - We follow the description of the BM in chap 5


% Local dynamics of the mean shear:
% - First the time dependence... U(t), Uc(t)
% - Second the effect of pressure gradient and viscosity (By equilibrium of forces...)-> validate with Laminar calculations!
\subsection{Local dynamics of the mean shear}
\begin{align}
\dot{u}=g\left(q,u\right)= \epsilon_{1} \left(U_{c}-u \right) + \epsilon_{2} \left(U-u \right)q \text{,}
\label{eq:loc_u}
\end{align}


% Local dynamics of turbulence intensity
% - First the time dependence: r!, U_c...
% - Second the effect of the inflection point: laminar profile is far from the parabolic profile. At intermediate amplitudes it is okay, but it doesnot seem so at higher amplitudes-
% - We opt for directly the inflection point ...
\subsection{Local dynamics of turbulence intensity}
\begin{align}
\dot{q}=f\left(q,u\right)=q \left[r+u-U_{c}-\left(r+\delta \right) \left(q -1 \right)^{2} \right]\text{.}
\label{eq:loc_q}
\end{align}


% Spatially extended model: 
% - just comment on the advection term of q...
% - Talk about the importance of the noise term. How do we scale it with Re...
\subsection{Spatially extended and stochastic model}
\begin{align}
\frac{\partial q}{\partial t}=-\left(u-\eta \right)\frac{\partial q}{\partial x} + f\left(q,u \right) + D\frac{\partial^{2} q}{\partial x^{2}} + \sigma \tau \left(t,x\right) q \text{,}
\label{eq:SPDE_q}
\end{align}

\begin{align}
\frac{\partial u}{\partial t}=-u\frac{\partial u}{\partial x} + g\left(q,u \right) + \frac{2}{Re}\frac{\partial^{2} u}{\partial x^{2}} \text{,}
\label{eq:PDE_u}
\end{align}


%\subsection{Numerical method}
%We integrate equations \ref{eq:SPDE_q} and \ref{eq:PDE_u} following \cite{barkley2015rise}. We discretise the second order derivatives with central finite differences of second order, and the first order derivatives with a first order upwind scheme. We integrate the system using an explicit Euler method, with a time step size $\Delta t=0.0025 D/U$. We consider a pipe of length $L=100D$ and a uniform grid spacing $\Delta x=0.5D$. The stochastic term is modeled as white gaussian noise in space and time. In figure ... we show a grid convergence study of our code.

%Later, and as part of this thesis a GPU version of the code was developed. The variables $q$ and $u$ are expanded in a trunceted Fourier mode series. The derivatives are then computed using Fourier. Two different integrators where considered, a two-step Adams Bashforth and a Runge Kutta of order 4. The details of the code can be found ¿? in appendix C?



%%%%%%%%%%%%%%%%%%%%%%%%%%%%%%%%%%%%%%%%%%%%%%%%%%%%%%%%%%%
% Results using the model
% - Parameter fitting and data base of comparison (why R0, eta)
% - Phase average comparison of model and DNS
% - Effect of ignoring new parameter... 
% - Effect of constant noise term...
% - Lessons learned from the model
\section{Results of the extended model}




%%%%%%%%%%%%%%%%%%%%%%%%%%%%%%%%%%%%%%%%%%%%%%%%%%%%%%%%%%%%%%%
% SECTION: Limitiations of my model
%   - Shape of the puff! (downstream front speed)
%   - Importance of noise + gamma parameter (some dynamics are not represented well enough)
%   - Over/underestimating lifetimes. 
\section{Limitations of the extended model}








% BIBLIOGRAPHY
\bibliographystyle{alpha}
\bibliography{sample}
\end{document}



% NEEDED MEDIA:


% NEEDED CITATIONS:
