\documentclass{article}

% PACKAGES
\usepackage[english]{babel}
\usepackage[letterpaper,top=2cm,bottom=2cm,left=3cm,right=3cm,marginparwidth=1.75cm]{geometry}
\usepackage{amsmath}
\usepackage{graphicx}
\usepackage[colorlinks=true, allcolors=blue]{hyperref}

% Title
\title{CHAPTER 5: Model for puffs and slugs in steady pipe flow}
\date{}
%%%%%%%%%%%%%%%%%%%%%%%%%%%%%%%%%%%%%%%%%%%%%%%%%%%%%%%%%%%%%%%%%%%%%%%%%%%%%%%%%%%%%%%%%%%%%%%%
% BEGIN DOCUMENT
%%%%%%%%%%%%%%%%%%%%%%%%%%%%%%%%%%%%%%%%%%%%%%%%%%%%%%%%%%%%%%%%%%%%%%%%%%%%%%%%%%%%%%%%%%%%%%%%
\begin{document}
\maketitle
% Introduction: 
% - Explain in two lines what the model does.
% - We show the model and we discuss how we interpret it.
% - We finally show the effect some parameters have in the model.

% Main idea and definition:
% - Main idea of the model: three steps
%   · interplay between mean shear and turbulence (nullclines?)
%   · advection/diffusion (excitable media!)
%   · noise

% Numerical method:

% Effect of parameters:
% - Fit to front speeds.
% - Fit to turbulence survival.
% - Show how other parameters donot have an effect









% BIBLIOGRAPHY
\bibliographystyle{alpha}
\bibliography{sample}
\end{document}
