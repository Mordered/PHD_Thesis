\documentclass{article}

% PACKAGES
\usepackage[english]{babel}
\usepackage[letterpaper,top=2cm,bottom=2cm,left=3cm,right=3cm,marginparwidth=1.75cm]{geometry}
\usepackage{amsmath}
\usepackage{graphicx}
\usepackage[colorlinks=true, allcolors=blue]{hyperref}
% Dani packages
\usepackage{xcolor}
\usepackage{xspace}
%\newcommand{\note}[1]{{\color{red} \textbf{#1}}}
\DeclareRobustCommand{\dm}[1]{{\color{blue}(\textbf{Dani}: \textit{#1}\xspace)}}

% Title
\title{CHAPTER 5: Model for puffs and slugs in steady pipe flow}
\date{}
%%%%%%%%%%%%%%%%%%%%%%%%%%%%%%%%%%%%%%%%%%%%%%%%%%%%%%%%%%%%%%%%%%%%%%%%%%%%%%%%%%%%%%%%%%%%%%%%
% BEGIN DOCUMENT
%%%%%%%%%%%%%%%%%%%%%%%%%%%%%%%%%%%%%%%%%%%%%%%%%%%%%%%%%%%%%%%%%%%%%%%%%%%%%%%%%%%%%%%%%%%%%%%%
\begin{document}
\maketitle
%%%%%%%%%%%%%%%%%%%%%%%%%%%%%%%%%%%%%%%%%%%%%%%%%%%%%%%%%%%%%%%%%%%%%%%%%%%%%%%%%%%%%%%%%%%%%%%%
% Introduction: 
% - Explain in two lines what the model does.
% - We show the model and we discuss how we interpret it.
% - We finally show the effect some parameters have in the model.
The model proposed by Barkley \cite{barkley2011modeling} (BM) is able to reproduce the behaviour of turbulent puffs and slugs in steady pipe flow. It considers only two one dimensional time dependent variables whose evolution is described by non-linearly coupled advection-diffusion-reaction partial differential equations. In this chapter we describe the model, we summarize the main ideas behind it and how we interpret them. We finally discuss the results of the model for steady pipe flow and the effect of its parameters, together with some limitations. \dm{La idea aqui es referenciar el capitulo (ahora 9) donde describo la extension a pulsatil.}
% We use this description in chapter 9, when we describe an adapted version of the model for pulsatile pipe flow.




%%%%%%%%%%%%%%%%%%%%%%%%%%%%%%%%%%%%%%%%%%%%%%%%%%%%%%%%%%%%%%%%%%%%%%%%%%%%%%%%%%%%%%%%%%%%%%%%
% Main idea and definition:
\section{Definition of the BM}
% - Main idea of the model: three steps
The BM considers the variables $q\left(x,t\right)$ and $u\left(x,t\right)$. Here $q$ represents the turbulence intensity and $u$ the state of the mean shear at time $t$ and on each axial location $x$ of the pipe. As a proxy to the mean shear, Barkley considers the centerline velocity of the pipe. Note that, in the case of steady pipe flow, the laminar profile is parabolic and its shear can be described with a single parameter. Therefore the centerline velocity, or the velocity at any radial position for that matter, is a good indicator of the state of the mean shear. When the centerline velocity is maximum, $u=U_{c}$, the flow is locally laminar as the value of the parabolic profile $U_{c}$ is recovered. Otherwise, when the flow is locally disturbed, the centerline velocity is $u < U_{c}$. The turbulence intensity $q$ is always either $q=0$, which models the case of no turbulence or laminar flow, or $q>0$. 

The main idea of the BM is the non-linear interaction between $u$ and $q$: the turbulence intensity $q$ takes advantage of the mean shear $u$ to grow, but the mean shear decreases when $q>0$. These dynamics are local in $x$, and can be modeled with a pair of non-linearly coupled ordinary differential equations. On top of this local interaction the model includes advection-diffusion and stochastic effects that, as we show below, are needed in order to describe the dynamics of puffs and slugs in statistically steady pipe flow. 

In this section we follow the derivation of the model presented in \cite{barkley2016}. We start with the local dynamics of $q$ and $u$. We then explain the spatially extended model, and we finally comment on the need to include a stochastic term to better capture the behaviour of localized turbulence in pipe flow.





%   · Turbulence intensity (effect of r parameter)
\subsection{Local dynamics of the turbulence intensity in the BM}
The variable $q$ represents the cross-section integral of turbulence energy on an axial location $x$. The flow is locally laminar when $q=0$ and turbulent if $q>0$. Its evolution equation is not derived from the Navier-Stokes equation. Instead the local dynamics of $q$ are described by the potential equation:

\begin{align}
\frac{\mathrm{d}q}{\mathrm{d}t}=\dot{q}= - \frac{\mathrm{d}V}{\mathrm{d}q}= f^{*} \left(q \right)=q \left[r-\left(r+\delta \right) \left(q -1 \right)^{2} \right]\text{,}
\label{eq:loc_q_0}
\end{align}

where $r$ is a control parameter and $\delta$ a model parameter. As we show below $r$ can be related with the Reynolds number of the actual pipe flow. The parameter $\delta$ loosely models the chance for transient growth of perturbations in pipe flow. The parameter is set to $\delta=0.1$. 

At low $r$ the potential $V$ has a single minima, meaning that, at low $r$ (Reynolds number) the local dynamics have a single equilibria at $q=0$ (laminar flow). At a certain $r>r_{1}$ the potential $V$ presents a global and a local minima. The global minima corresponds to $q=0$, and the local minima to a certain $q>0$. At this $r$ the local dynamics are unconditionally stable for $q=0$, and metastable for $q>0$ (localized turbulence). As $r$ further increases to $r>r_{2}$ $q=0$ becomes a local minima of the system, while $q>0$ the global one. In this case, the laminar profile is metastable, whereas the turbulent profile is unconditionally stable. When compared with the actual behaviour of puffs and slugs in pipe flow, $r_{1}$ corresponds to the first Reynolds number at which sufficiently big perturbations are able to grow into puffs that subsequently decay. On the other hand $r_{2}$ corresponds to the first Reynolds number at which turbulence spreads, or in other words, when puffs first start to elongate to become slugs. 

A similar equation to eq~\ref{eq:loc_q_0}, was first proposed by Landau CITE, to model turbulence transition as a phase transition. However, it failed to describe the localization of puffs for a wide range of $Re$, and their decay and split mechanisms. As we show below, in order to capture this behavior, one needs to consider an additional variable, the mean shear.






%   · Mean shear 
\subsection{Local dynamics of the mean shear}
The variable $u$ corresponds to the centerline velocity of the pipe flow, as a proxy of the state of the local mean shear. When the mean shear is not disturbed, $u$ is maximum and equal to the centerline velocity of the laminar (parabolic) profile $u=U_{c}$. When the mean shear is disturbed, $u<U_{c}$. The model assumes that the most disturbed mean shear possible corresponds to a plug like profile, where the centerline velocity equals the bulk velocity $u=\bar{U}$. 

The ordinary differential equation that describes the evolution of $u$ is not derived from the Navier--Stokes equation. Instead it is modeled as:
\begin{align}
\dot{u}=g\left(q,u\right)= \epsilon_{1} \left(U_{c}-u \right) + \epsilon_{2} \left(\bar{U}-u \right)q \text{,}
\label{eq:loc_u}
\end{align}

where $\epsilon_{1}$ and $\epsilon_{2}$ are model parameters. 

Accroding to equation~\ref{eq:loc_u}, if the flow locally has no turbulence intensity $q=0$, and $u<U_{c}$, the centerline velocity increases until it reaches the centerline velocity $U_{c}$ of the laminar (parabolic) profile. The characteristic time of this relaminarization process is set by the model parameter $\epsilon_{1}$. According to Barkley, this parameter should be inversely proportional to the Reynolds number \cite{barkley2016}, but it is always set constant. 

When the flow is locally turbulent $q>0$ the centerline velocity decreases due to the second term in the RHS of equation~\ref{eq:loc_u}, with a lower bound at $u=\bar{U}$. The parameter $\epsilon_{2}$ describes the effect the turbulence intensity has on the mean shear. 


%   · Local interactions
\subsection{Non-linear interactions}
The evolution of $u$, eq.~\ref{eq:loc_u}, considers the effect $q$ has on $u$. However the equation for $q$, eq.~\ref{eq:loc_q_0}, does not include a coupling to the mean shear. Barkley corrects this by changing eq.~\ref{eq:loc_q_0} to: 
\begin{align}
\dot{q}=f\left(q,u\right)=q \left[r+u-U_{c}-\left(r+\delta \right) \left(q -1 \right)^{2} \right]\text{.}
\label{eq:loc_q}
\end{align}

The idea is that the growth of $q$ is maximum when $u \equiv  2$, or when the mean profile is parabolic (laminar profile). This is in line with direct numerical simulations and experiment results, where a blunter profile has been linked with a decrease in turbulence lifetime CITE.

The non-linearly coupled ordinary differential equations eq.~\ref{eq:loc_q} and eq.~\ref{eq:loc_u} describe the local dynamics of $q$ and $u$. For fixed $\delta$, $\epsilon$ and $\epsilon_{2}$, see table ... , they only depend on the control parameter $r$ that sets the equilibrium points of the equations $\dot{q}=\dot{u}=0$. In figure ... we plot the position of the equilibrium points and the phase diagram of the system for different $r$. Depending on $r$ the system can either have one equilibrium point (laminar) or two (laminar and turbulent).



%   · advection/diffusion (front speeds)
\subsection{Spatially extended model}
Alone, the local interactions of the model are not able to describe the behaviour of puffs and slugs in pulsatile pipe flow. Puffs and slugs are localized structures that are advected and diffussed in the pipe. Therefore the model needs to consider spatially extended effects: advection and diffusion. 

In the case of the turbulence intensity the Barkley model equation is
\begin{align}
\frac{\partial q}{\partial t}=-\left(u-\eta \right)\frac{\partial q}{\partial x} + f\left(q,u \right) + D\frac{\partial^{2} q}{\partial x^{2}} \text{,}
\label{eq:PDE_q}
\end{align}
being $\eta$ and $D$ two new model parameters. $D$ represents the diffusion coefficient of $q$ on the axial direction. The model assumes that $q$ is advected at velocity $u$, corrected with the parameter $\eta$

In the case of the centerline velocity, the equation reads
\begin{align}
\frac{\partial u}{\partial t}=-u\frac{\partial u}{\partial x} + g\left(q,u \right) + \frac{2}{Re}\frac{\partial^{2} u}{\partial x^{2}} \text{,}
\label{eq:PDE_u}
\end{align}
where usually the last term is ignored. According to Barkley, the addition of this term does not change significantly the results.





%    · about puffs and slugs in the BM
\subsection{About puffs and slugs in the BM}
Equations \ref{eq:PDE_q} and \ref{eq:PDE_u} together with \ref{eq:loc_q} and \ref{eq:loc_u}, define the deterministic version of the original Barkley Model. When its 7 parameters are correctly fitted, the model is able to perfectly capture the front speeds of puffs and slugs in pipe flow. More importantly the model indicates what kind of dynamical systems a puff and a slug are.

At $r=r_{1}$ (figure a) the system has a single point of equilibrium at $u=q=0$ (laminar flow). However, at this $r$, when sufficiently perturbed, the system can have $q>0$ during a certain time span before returning to $q=0$. This corresponds to an excitable media, a dynamical system that has only one stable equilibrium point, but that can be perturbed to an excitable state. Once excited the system takes a certain time to relax again to its equilibrium condition, and to be able to be excited again (refractory time, length). This models the behaviour of puffs, that can be triggered by exciting the laminar profile. Once a puff has been triggered one needs to wait a certain refractory length to trigger a second puff. 

At $r=r_{2}>r_{1}$ the system has two stable equilibrium points at $q=u=0$ (laminar flow) and $q>0$ $u<2$ (turbulent flow). This corresponds to a bi-stable system, and models the behaviour of slugs where a region of turbulent flow is stable and expands in a quiescent laminar one. The model is able to distinguish between weak $r_{3}>r>r_{2}$ and strong slugs $r>r_{3}$. 





%   · noise
\subsection{Stochastic model}
Puffs and slugs are chaotic structures. Puffs can (randomly) decay, or split, and slugs present intermittent behaviours, with sudden changes of front speed or relaminarization in some section of their core. The PDEs in equations \ref{eq:PDE_q} and \ref{eq:PDE_u} donot result in chaotic dynamics. In order to model decay, split or even intermittency, A stochastic term is included in the model as:
\begin{align}
\frac{\partial q}{\partial t}=-\left(u-\eta \right)\frac{\partial q}{\partial x} + f\left(q,u \right) + D\frac{\partial^{2} q}{\partial x^{2}} + \sigma \tau \left(t,x\right) q \text{,}
\label{eq:SPDE_q}
\end{align}
being $\tau$ white gaussian noise in space and time, and $\sigma$ the strength of the noise. 

Equations \ref{eq:SPDE_q} and \ref{eq:PDE_u} together with \ref{eq:loc_q} and \ref{eq:loc_u}, define the original Barkley Model. The model includes 7 parameters and one control parameter.



%%%%%%%%%%%%%%%%%%%%%%%%%%%%%%%%%%%%%%%%%%%%%%%%%%%%%%%%%%%%%%%%%%%%%%%%%%%%%%%%%%%%%%%%%%%%%%%
% Results using the model
\section{Results of the model}
By fitting the model parameters, see table ..., the BM can perfectly reproduce the front speeds of puffs and slugs for a huge number of $Re$. The front speeds measure the velocity at which the puffs and slugs are advected or elongate in pipe flow. In the case of the puff the upstream $c_{u}$ and downstream velocity $c_{d}$ are equal, while for the case of the slug $c_{d}>c_{u}$. The case of a strong slug also satisfies $c_{d}\approx 2U-c_{u}$. Recent results suggest that the model gives a perfect estimation of $c_{u}$ and $c_{d}$ at $Re$ as high as ... CITE. With the inclusion of the stochastic term the model also approximates the lifetimes of puffs and slugs. In this section we show the robustness of the model to parameter changes, and comment on limitations of the model that, as we will show later, are critical for the correct extension of the model to pulsatile pipe flow. 

% - Show how other parameters donot have an effect
\subsection{Effect of changing parameters}
In its original paper, Barkley shows the robustness of his model after changing its parameters. In the extended model of pulsatile pipe flow, we select a slighly different set of parameters to those in table --- In this subsection we show the little effect they have on the front speeds of puffs in steady pipe flow. In figure ... we show both the front speeds using the original fit of Barkley, and in the front speeds after changing the parameters to a new set ... The differences between the two results are remarkably small.



%%%%%%%%%%%%%%%%%%%%%%%%%%%%%%%%%%%%%%%%%%%%%%%%%%%%%%%%%%%%%%%%%%%%%%%%%%%%%%%%%%%%%%%%%%%%%%%
% - Limitiations of the original model 
%   - Shape of the puff! (downstream front speed)
%   - Lifetimes (shape) and the importance of non-linearities
\section{Limitations of the model}
Despite being able to perfectly capture puffs and slugs front speeds, the model has some limitations that are important to mention, specially before adapting the model to the pulsatile case.

Firstly, the model doesnot capture the actual shape of slugs in steady pipe flow. See in figure ... a comparison of slugs in full DNS and on the model. The model overestimates the amplitude of $q$ in the core of the slug. In the case of the DNS there is a clear difference between the turbulence magnitude in the upstream front and on the core. This was already noted by Barkley CITE. This is not critical for the behaviour of the slugs in the model, when compared with the behaviour in DNS. It causes qualitative differences when comparing the results from DNS and the model. 

Secondly, the model doesnot capture the whole non-linear dynamics of puffs and slugs in pipe flow. Even after adding the stochastic term, the model doesnot return the same exponential scaling of puffs lifetimes and splitting probabilities. This means that even the original model is not able to capture the full spectra of dynamics in the steady case. One possible solution could be to better define the parameter $\sigma$ or the noise term $\gamma$. For starters, the parameter $\sigma$ could scale up with $Re$ since higher Reynolds number means more intense chaotic dynamics.

Thirdly, and related with the previous idea, the model does not capture the intermittent behaviour of puffs, specially in the $Re$ regime between $2250 \lesssim Re \lesssim 2500$. In this regime, according to the BM, puffs elongate into slugs. However, in full DNS, puffs  in this regime are observed to elongate and split in a stochastic manner. At these $Re$ the flow usually reaches a highly intermitent state, where localized turbulent patches coexist with laminar flow patches. This behaviour is only observed in the model, if $\sigma$ is increased.




%%%%%%%%%%%%%%%%%%%%%%%%%%%%%%%%%%%%%%%%%%%%%%%%%%%%%%%%%%%%%%%%%%%%%%%%%%%%%%%%%%%%%%%%%%%%%%%%
% BIBLIOGRAPHY
\bibliographystyle{alpha}
\bibliography{sample}
\end{document}



% NEEDED MEDIA:
% 0. 3 plots of the potential
% 1. Mean profile + Phase-space diagram + Equilibrium points Maybe compare with DNS? (BAOFANG)
% 2. Table of fit parameters
% 3. Show the qualitative behaviour of puffs between 2250 Re 2500


% NEEDED CITATIONS:
% Landau? the potential equation?
% Kuhnen plug profile
% New paper DUO and SONG about front speeds


% Numerical method: FOR CHAPTER OF PULSATILE CASE
%\subsection{Numerical method}
%We integrate equations \ref{eq:q_BM} and \ref{u_BM} following \cite{barkley2015rise}. We discretise the second order derivatives by central finite differences of second order, and the first order derivatives with a first order upwind scheme. We integrate the system using an explicit Euler method, with a time step size $\Delta t=0.0025 D/U$. We consider a pipe of length $L=100D$ and a uniform grid spacing $\Delta x=0.5D$. The stochastic term is modeled as white gaussian noise in space and time. In figure ... we show a grid convergence study of our code.

%Later, and as part of this thesis a GPU version of the code was developed. The variables $q$ and $u$ are expanded in a trunceted Fourier mode series. The derivatives are then computed using Fourier. Two different integrators where considered, a two-step Adams Bashforth and a Runge Kutta of order 4. The details of the code can be found ¿? in appendix C?