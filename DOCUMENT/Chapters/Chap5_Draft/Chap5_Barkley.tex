\documentclass{article}

% PACKAGES
\usepackage[english]{babel}
\usepackage[letterpaper,top=2cm,bottom=2cm,left=3cm,right=3cm,marginparwidth=1.75cm]{geometry}
\usepackage{amsmath}
\usepackage{graphicx}
\usepackage[colorlinks=true, allcolors=blue]{hyperref}

% Title
\title{CHAPTER 5: Model for puffs and slugs in steady pipe flow}
\date{}
%%%%%%%%%%%%%%%%%%%%%%%%%%%%%%%%%%%%%%%%%%%%%%%%%%%%%%%%%%%%%%%%%%%%%%%%%%%%%%%%%%%%%%%%%%%%%%%%
% BEGIN DOCUMENT
%%%%%%%%%%%%%%%%%%%%%%%%%%%%%%%%%%%%%%%%%%%%%%%%%%%%%%%%%%%%%%%%%%%%%%%%%%%%%%%%%%%%%%%%%%%%%%%%
\begin{document}
\maketitle
% Introduction: 
% - Explain in two lines what the model does.
% - We show the model and we discuss how we interpret it.
% - We finally show the effect some parameters have in the model.
The model proposed by Barkley \cite{barkley2011modeling}, known as the Barkley Model, BM, is able to reproduce the behaviour of turbulent puffs and slugs in steady pipe flow. It consider two one dimensional time dependent variables whose evolution is described by coupled, non-linear advection-diffusion-reaction partial differential equations. In this chapter we describe the model, we summarize the main ideas behind it and how we interpret them. We finally discuss the results of the model and the effect of its parameters. 


% Main idea and definition:
% - Main idea of the model: three steps
%   · interplay between mean shear and turbulence (nullclines?)
%   · advection/diffusion (excitable media!)
%   · noise


% Numerical method:



% Effect of parameters:
% - Fit to front speeds.
% - Fit to turbulence survival.
% - Show how other parameters donot have an effect









% BIBLIOGRAPHY
\bibliographystyle{alpha}
\bibliography{sample}
\end{document}
